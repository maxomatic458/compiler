\section{Fazit}
    \subsection{Ergebnisse}
        In meiner Sprache ist es möglich einfache, aber auch komplexe Programme zu schreiben.
        Vor allem durch Generics und Operatorüberladung lassen sich komplexere Probleme einfacher lösen.

        Es ist möglich Programme direkt als ausführbare Datei zu kompilieren, oder 
        sie in LLVM IR kompilieren. Fehlermeldungen sind in den meisten Fällen sehr ausführlich und hilfreich.

        Weil sowohl ein eigener Lexer als auch ein eigener Parser implementiert wurde,
        ist der Compiler recht flexibel gestaltet und kann leicht erweitert oder angepasst werden.

        Und während der Entwicklung konnte ich viel über Programmiersprachen und Compilerarchitektur lernen.

    \subsection{Verbeserungsmöglichkeiten}
        Es fehlen einige Features, die normalerweise in einer modernen Programmiersprache vorhanden sind:
        \begin{itemize}
            \item Multithreading ist nicht möglich
            \item Es gibt keine Bitoperationen
            \item Es gibt nur öffentliche Funktionen und Klassen
            \item Es gibt keine Namespaces
        \end{itemize}

        Folgende Dinge könnten verbessert werden:
        \begin{itemize}
            \item Der IR Code wird bis zum Schluss in einer Liste gespeichert. 
                Bei größeren Programmen wäre es sinnvoller direkt bei der Generierung den IR Code in die Datei zu schreiben. 
            \item Der Codegenerator ist single-threaded, 
                Multithreading könnte z.B. bei dem Importieren von Dateien oder bei der Generierung des IR Codes implementiert werden.
            \item Eventuell könnte ich mir noch einen Namen für meine Sprache ausdenken.
            \item Ein Rust ähnliches Trait-Bound System wäre sinnvoll, um Generics vollständig Typensicher zu machen.
            \item Eine automatische Freigabe des Heap-Speichers ähnlich wie in Rust wäre sinnvoll (Ownership-System).
        \end{itemize}